\documentclass[11pt]{article}

\usepackage{mathtools}

\usepackage[noend]{algpseudocode}
\usepackage{algorithm}
\usepackage{algorithmicx}
\newcommand*\Let[2]{\State #1 $\gets$ #2}
\algrenewcommand\algorithmicrequire{\textbf{Input:}}
\algrenewcommand\algorithmicensure{\textbf{Input:}}
\newcommand\numberthis{\addtocounter{equation}{1}\tag{\theequation}}

\newcommand{\epi}{\mathrm{epi}}
\newcommand{\Rea}{{\Bbb R}}
\newcommand{\Int}{{\Bbb Z}}
\newcommand{\Rat}{{\Bbb Q}}
\newcommand{\Cmp}{{\Bbb C}}
\newcommand{\Nat}{{\Bbb N}}
\newcommand{\NN}{\mathbb{N}}
\newcommand{\RR}{\mathbb{R}}
\newcommand{\oRR}{\overline{\RR}}
\newcommand{\argmax}{\mathrm{argmax}}
\newcommand{\argmin}{\mathrm{argmin}}
\newcommand{\interior}{\mathrm{int}}
\newcommand{\Fix}{\mathrm{Fix}}
\newcommand{\dom}{\mathrm{dom}}
\newcommand{\graph}{\mathrm{gph}}
\DeclarePairedDelimiter{\dotp}{\langle}{\rangle}


\setlength{\oddsidemargin}{0in}
\setlength{\evensidemargin}{0in}
\setlength{\textwidth}{6.5in}
\setlength{\topmargin}{-.25in}
\setlength{\textheight}{8.5in}
%\setlength{\voffset}{0in}
%\setlength{\headheight}{0in}
%\setlength{\headsep}{0in}
%\setlength{\topskip}{0in}

\newtheorem{definition}{Definition}
\newtheorem{remark}{Remark}
\newtheorem{theorem}{Theorem}
\newtheorem{lemma}[theorem]{Lemma}
\newtheorem{corollary}[theorem]{Corollary}
\newtheorem{proposition}[theorem]{Proposition}
\newtheorem{claim}[theorem]{Claim}
\newtheorem{observation}{Observation}
\newtheorem{fact}{Fact}
\newtheorem{example}{Example}
\newtheorem{exercise}{Exercise}
\newtheorem{notation}{Notation}




\newenvironment{proof}{\noindent{\bf Proof:} \hspace*{1em}}{
    \hspace*{\fill} $\Box$ }
\newenvironment{proof_of}[1]{\noindent {\bf Proof of #1:}
    \hspace*{1em} }{\hspace*{\fill} $\Box$ }
\newenvironment{proof_claim}{\begin{quotation} \noindent}{
    \hspace*{\fill} $\diamond$ \end{quotation}}


\newcommand{\handout}[5]{
   \renewcommand{\thepage}{#1-\arabic{page}}
   \noindent
   \begin{center}
   \framebox{
      \vbox{
    \hbox to 6.5in { {\bf ORIE 6300 Mathematical Programming I} \hfill #2 }
       \vspace{4mm}
       \hbox to 6.5in { {\Large \hfill #5  \hfill} }
       \vspace{2mm}
       \hbox to 6.5in { {\it #3 \hfill #4} }
      }
   }
   \end{center}
   \vspace*{4mm}
}

\newcommand{\lecture}[4]{\handout{#1}{#2}{Lecturer:
#3}{Scribe: #4}{Lecture #1}}
\newcommand{\problemset}[4]{\handout{#1}{#2}{}{Due Date: #4}{Problem Set #3}}
\newcommand{\problemsetsoln}[3]{\handout{#1}{#2}{}{}{Problem Set #3 Solutions}}
\newcommand{\exam}[3]{\handout{#1}{#2}{}{Due Date: #3}{Take-Home Final Exam}}
\newcommand{\examsoln}[2]{\handout{#1}{#2}{}{}{Take-Home Final Exam Solutions}}
\newcommand{\recitation}[4]{\handout{#1}{#2}{Lecturer:
#3}{Topic: #4}{Recitation #1}}


\newcommand{\dpw}{David P.\ Williamson}

\newenvironment{alglist}{\begin{list}{}{\setlength{\leftmargin}{1.5cm}
\setlength{\rightmargin}{0cm}\setlength{\itemsep}{1ex}\setlength{\parsep}{1ex}}}{\end{list}}

\newcommand{\problem}[3]
{\fbox{\parbox{6in}{{\bf #1}\begin{itemize}\item{\bf Input:} {#2} \item{\bf Goal:} {#3}\end{itemize}}}}
%\newcommand{\algorithm}[2]
%{\fbox{\fbox{\begin{minipage}{6in}
%{\bf #1}\vspace*{.1cm}\hrule
%\begin{tabbing}
%99\qquad\=\qquad\=\qquad\=\qquad\=\qquad\=\qquad\=\kill
%#2
%\end{tabbing}
%\end{minipage}}}}

\usepackage{amsmath, latexsym, epsf,amssymb}

\newcommand{\RR}{\mathbb{R}}
\begin{document}

\problemset{2}{September 1, 2016}{2}{September 8, 2016}

\begin{enumerate}

\item Give an example of a primal-dual pair for which both the
primal and dual are infeasible, and demonstrate that they are
infeasible.  Use a matrix $A \in \Re^{1 \times 1}$.

\item Let $\mathbf{1} \in \RR^n$ be the vector of all ones. Consider the set of doubly stochastic matrices
\begin{align*}
X = \{A \in \RR^{n\times n} \mid  A \geq 0 \text{ (entrywise)}, \mathbf{1}^TA = \mathbf{1}^T \text{ and } A\mathbf{1} = \mathbf{1}\}.
\end{align*}
Prove that $X$ is convex and a polytope. Show that the set of extreme points of $X$ is exactly the set of permutation matrices $\mathcal{P}$, i.e., those binary matrices $P \in \RR^{n \times n}$ that have exactly one entry equal to $1$ in each row and each column and $0$s elsewhere. 

\item Let $C \subseteq \RR^n$ be a closed convex set and let $y \in \RR^n$ be a vector. 
\begin{enumerate}
\item Show that $f : C \rightarrow \RR^n$, defined by
\begin{align*}
\left(\forall x \in C\right)\quad  f(x)  = \frac{1}{2}\|x - y\|^2
\end{align*}
has a unique minimizer $x^\ast \in C$. (Hint: recall that in lecture 4 we showed $f$ has at least one minimizer in $C$.) 
\item Show that 
\begin{align*}
\left(\forall z \in C\right) \qquad \|x^\ast - z\|^2 + \|x^\ast - y\|^2 \leq \| y - z\|^2.
\end{align*}
(Hint: recall from the proof of the separating hyperplane theorem, we had, for some $b \in \RR$, that $\left(\forall z \in C\right) \; (y- x)^T z < b < (y - x)^T y.$) 
\item Conclude that the \textit{projection mapping} $P_C : \RR^n \rightarrow \RR^n$, defined by
\begin{align*}
\left(\forall y \in \RR^n\right) \qquad P_C(y) = \text{the unique minimizer $x^\ast \in C$ of $f(x) = \frac{1}{2} \|x - y\|^2$},
\end{align*}
is well-defined and
\begin{align*}
\left(\forall z \in C\right),\left(\forall y \in \RR^n\right) \qquad \|P_C(y) - z\|^2 + \|P_C(y) - y\|^2 \leq \|y - z\|^2.
\end{align*}
\end{enumerate}

\item Suppose that you are given a feasible solution $\bar x$ of value
$\bar \gamma$ to the problem $\max (c^T x: A x \le b)$. Give a method that either demonstrates that the feasible
region is unbounded (i.e., there is a point $x$ and direction $y$ such that
$x+\lambda y$ is feasible for all $\lambda >0$)
or that finds a vertex $\tilde x$ of the feasible region with objective
value $c \tilde x \ge \bar \gamma$. Your method should not use general
purpose linear programming algorithms (like the simplex method).
(Hint: some of the discussion of the equivalence of bounded polyhedra and polytopes, as well as the equivalence of
extreme points, vertices, and basic feasible solutions, might
prove useful.)

\end{enumerate}
\end{document}
