\documentclass[11pt]{article}

\usepackage{mathtools}

\usepackage[noend]{algpseudocode}
\usepackage{algorithm}
\usepackage{algorithmicx}
\newcommand*\Let[2]{\State #1 $\gets$ #2}
\algrenewcommand\algorithmicrequire{\textbf{Input:}}
\algrenewcommand\algorithmicensure{\textbf{Input:}}
\newcommand\numberthis{\addtocounter{equation}{1}\tag{\theequation}}

\newcommand{\epi}{\mathrm{epi}}
\newcommand{\Rea}{{\Bbb R}}
\newcommand{\Int}{{\Bbb Z}}
\newcommand{\Rat}{{\Bbb Q}}
\newcommand{\Cmp}{{\Bbb C}}
\newcommand{\Nat}{{\Bbb N}}
\newcommand{\NN}{\mathbb{N}}
\newcommand{\RR}{\mathbb{R}}
\newcommand{\oRR}{\overline{\RR}}
\newcommand{\argmax}{\mathrm{argmax}}
\newcommand{\argmin}{\mathrm{argmin}}
\newcommand{\interior}{\mathrm{int}}
\newcommand{\Fix}{\mathrm{Fix}}
\newcommand{\dom}{\mathrm{dom}}
\newcommand{\graph}{\mathrm{gph}}
\DeclarePairedDelimiter{\dotp}{\langle}{\rangle}


\setlength{\oddsidemargin}{0in}
\setlength{\evensidemargin}{0in}
\setlength{\textwidth}{6.5in}
\setlength{\topmargin}{-.25in}
\setlength{\textheight}{8.5in}
%\setlength{\voffset}{0in}
%\setlength{\headheight}{0in}
%\setlength{\headsep}{0in}
%\setlength{\topskip}{0in}

\newtheorem{definition}{Definition}
\newtheorem{remark}{Remark}
\newtheorem{theorem}{Theorem}
\newtheorem{lemma}[theorem]{Lemma}
\newtheorem{corollary}[theorem]{Corollary}
\newtheorem{proposition}[theorem]{Proposition}
\newtheorem{claim}[theorem]{Claim}
\newtheorem{observation}{Observation}
\newtheorem{fact}{Fact}
\newtheorem{example}{Example}
\newtheorem{exercise}{Exercise}
\newtheorem{notation}{Notation}




\newenvironment{proof}{\noindent{\bf Proof:} \hspace*{1em}}{
    \hspace*{\fill} $\Box$ }
\newenvironment{proof_of}[1]{\noindent {\bf Proof of #1:}
    \hspace*{1em} }{\hspace*{\fill} $\Box$ }
\newenvironment{proof_claim}{\begin{quotation} \noindent}{
    \hspace*{\fill} $\diamond$ \end{quotation}}


\newcommand{\handout}[5]{
   \renewcommand{\thepage}{#1-\arabic{page}}
   \noindent
   \begin{center}
   \framebox{
      \vbox{
    \hbox to 6.5in { {\bf ORIE 6300 Mathematical Programming I} \hfill #2 }
       \vspace{4mm}
       \hbox to 6.5in { {\Large \hfill #5  \hfill} }
       \vspace{2mm}
       \hbox to 6.5in { {\it #3 \hfill #4} }
      }
   }
   \end{center}
   \vspace*{4mm}
}

\newcommand{\lecture}[4]{\handout{#1}{#2}{Lecturer:
#3}{Scribe: #4}{Lecture #1}}
\newcommand{\problemset}[4]{\handout{#1}{#2}{}{Due Date: #4}{Problem Set #3}}
\newcommand{\problemsetsoln}[3]{\handout{#1}{#2}{}{}{Problem Set #3 Solutions}}
\newcommand{\exam}[3]{\handout{#1}{#2}{}{Due Date: #3}{Take-Home Final Exam}}
\newcommand{\examsoln}[2]{\handout{#1}{#2}{}{}{Take-Home Final Exam Solutions}}
\newcommand{\recitation}[4]{\handout{#1}{#2}{Lecturer:
#3}{Topic: #4}{Recitation #1}}


\newcommand{\dpw}{David P.\ Williamson}

\newenvironment{alglist}{\begin{list}{}{\setlength{\leftmargin}{1.5cm}
\setlength{\rightmargin}{0cm}\setlength{\itemsep}{1ex}\setlength{\parsep}{1ex}}}{\end{list}}

\newcommand{\problem}[3]
{\fbox{\parbox{6in}{{\bf #1}\begin{itemize}\item{\bf Input:} {#2} \item{\bf Goal:} {#3}\end{itemize}}}}
%\newcommand{\algorithm}[2]
%{\fbox{\fbox{\begin{minipage}{6in}
%{\bf #1}\vspace*{.1cm}\hrule
%\begin{tabbing}
%99\qquad\=\qquad\=\qquad\=\qquad\=\qquad\=\qquad\=\kill
%#2
%\end{tabbing}
%\end{minipage}}}}

\usepackage{amsmath, latexsym,amssymb}
\newcommand\numberthis{\addtocounter{equation}{1}\tag{\theequation}}

\begin{document}

\newcommand{\Tcp}{T_{\mathrm{CP}}}
\newcommand{\Fix}{\mathrm{Fix}}
\problemset{7}{October 27, 2016}{7}{November 3, 2016}

\begin{enumerate}

\item Let $\gamma$ and $\tau$ be positive real numbers \textbf{that satisfy} $\gamma \tau < \frac{1}{\|A\|^2}$. Consider the Chambolle-Pock operator
\begin{align*}
T_{\mathrm{CP}} : \RR^m \times \RR^n &\rightarrow \RR^m \times \RR^n\\
T_{\mathrm{CP}} \begin{bmatrix} 
y \\ 
x
\end{bmatrix} &:= 
\begin{bmatrix} 
y - \gamma  \left(Ax - b\right) \\
\max\{ x + \tau \left(A^T(y - 2\gamma(Ax - b)) - c\right), 0\}
\end{bmatrix},
\end{align*}
In this exercise, we're going to prove that $\Tcp$ is firmly-nonexpansive in a Mahalanobis norm $\|x\|_Q$, i.e., 
\begin{align*}
&\left(\forall z_1 \in \RR^{m+n}\right), \left(\forall z_2 \in \RR^{m+n}\right) \\
&\hspace{20pt}\|\Tcp z_1 - \Tcp z_2\|_Q^2 \leq \|z_1 - z_2\|_Q^2 - \|(z_1 - \Tcp z_1) - (z_2 - \Tcp z_2)\|_Q^2, \numberthis\label{eq:firmnonexpansive}
\end{align*}
where 
$$
Q = \begin{bmatrix}
\frac{1}{\gamma} I  &  - A \\
- A^T & \frac{1}{\tau} I 
\end{bmatrix}.
$$
~\\
Define the \textit{set-valued} mapping $M : \RR^{m+n} \rightarrow 2^{\RR^{m + n}}$: for all $z = (y, x) \in \RR^{m+n}$,
\begin{align*}
Mz := \{-b\} \times (c + N_{\RR^m_{\geq 0}}(x)) + \begin{bmatrix} 
0 & A \\
-A^T & 0 
\end{bmatrix}\begin{bmatrix} 
y \\
x
\end{bmatrix}.
\end{align*}
\begin{enumerate}
\item \label{item:a}Let $z =  (y,x) \in \RR^{m + n}$. Show that
\begin{align*}
Q \left(z - \Tcp z\right) \in M \Tcp z
\end{align*}
(\textbf{Hint:} use the projection inclusion formula $x - P_C(x) \in N_{C}(P_C(x))$).
\item Let $z_1 =  (y_1,x_1) \in \RR^{m + n}$ and $z_2 = (y_2, x_2) \in \RR^{m+n}$. Show that 
\begin{align*}
&\left(\forall u_1 \in Mz_1\right), \left(\forall u_2 \in M z_2 \right) \qquad \langle z_1 - z_2, u_1 - u_2\rangle \geq 0
\end{align*}
(this condition states that $M$ is a \textit{monotone operator}). Using Part~\ref{item:a}, conclude that 
\begin{align*}
\langle  (z_1 - \Tcp z_1) - (z_2 - \Tcp z_2), \Tcp z_1 - \Tcp z_2\rangle_Q \geq 0.
\end{align*}
where for all $z, z' \in \RR^{m+n}$, we have $\langle z, z' \rangle_Q  = \langle Qz, z'\rangle$.
\item Prove~\eqref{eq:firmnonexpansive}.
\end{enumerate}

\item This exercise shows that solving a system of linear inequalities is essentially as hard as solving an LP.

Let $P(A, b) = \{x \mid Ax = b, x \geq 0\}$. Suppose that $x^\ast$ is a minimizer of $\min_{x \in P(A, b)} c^T x$. Let $x_0 \in \RR^n$ and for all $\gamma > 0$, define 
$$
x_\gamma = P_{P(A, b)} (x_0 - \gamma c).
$$
Prove that 
\begin{align*}
\langle c, x_\gamma\rangle \leq \langle c, x^\ast\rangle  + \frac{1}{2\gamma}\|x_0 - x^\ast\|^2.
\end{align*}
For which $\gamma > 0$ is $x_\gamma$ an $\varepsilon$-accuracy solution of the LP? (Recall that $x$ is an $\varepsilon$-accuracy solution if it is feasible and $\langle c, x\rangle < \langle c, x^\ast \rangle + \varepsilon.$) 

\item In this exercise, we learn how to parallelize the Douglas-Rachford Splitting (DRS) algorithm and the Method of Alternating Projections (MAP) through the \textit{product-space trick.}

Consider $l$ closed convex sets $C_1, \ldots, C_l \subseteq \RR^r$. Assume that $C_1 \cap \ldots\cap C_l \neq \emptyset.$ Define $C = C_1\times \cdots \times C_l$. Define the \textit{diagonal} vector subspace $V \subseteq \RR^{rl}$:
\begin{align*}
V : = \{(x_1, \ldots, x_l)  \in \RR^{rl} \mid \left(\forall i \right)\;  x_i \in \RR^r, x_1 = x_2 = \cdots= x_l\}.
\end{align*} 
\begin{enumerate}
\item Given $z \in\RR^{rl}$, compute $P_Vz$ and determine $\Fix(P_V)$.
\item Given $z \in \RR^{rl}$, compute $P_{C}z$ and determine $\Fix(P_{C})$.
\item Determine $\Fix(P_V P_C)$ and $\Fix\left(\frac{1}{2}(2P_V - I) \circ (2P_C - I) + \frac{1}{2} I\right)$
\item Consider the primal-dual pair of linear programs
\begin{align*}
\min\{ c^Tx \mid Ax = b, x \geq 0\} && \text{and} && \max\{b^T y \mid A^T y\leq  c\},
\end{align*}
and assume that there exists a primal-dual optimal solution, e.g., $(x^\ast, y^\ast) \in \RR^{n+m}$. Define
\begin{align*}
D := \begin{bmatrix}
A & 0 & 0 \\
0 & A^T & I \\
c^T & - b^T & 0 
\end{bmatrix}
&& \text{and} && d := \begin{bmatrix} 
b \\
c \\ 
0
\end{bmatrix}.
\end{align*}
Note that $Dz = d$ has at least one solution because the LPs are solvable. Let $l = n + m +2$ and define
\begin{align*}
C_l := \left\{ \begin{bmatrix} x \\ y \\ s \end{bmatrix} \mid  x \geq 0, s \geq 0\right\}.
\end{align*}
Provide $l - 1$ sets $C_1, \ldots, C_{l-1} \subseteq \RR^{m + 2n}$ such that (1) $\{ z \mid Dz = d\} = C_1 \cap \ldots \cap C_{l-1}$ and (2) for each $i = 1, \ldots, l-1$, the set $C_i$ is defined purely in terms of the $i$th rows of $D$ and $d$.

As before, define $V \subseteq \RR^{l(m+2n)}$  and $C := C_1 \times \ldots \times C_l$. Given $z \in \RR^{l(m + 2n)}$ compute  $P_V P_C (z)$.  What is the biggest computational drawback of this approach? Are there other ways to split $\{z \mid Dz = d\}$ into fewer sets? (There is no single correct answer.) 
\end{enumerate}
\end{enumerate}

\end{document}
