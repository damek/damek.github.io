\documentclass[11pt]{article}
\usepackage{mathtools}

\usepackage[noend]{algpseudocode}
\usepackage{algorithm}
\usepackage{algorithmicx}
\newcommand*\Let[2]{\State #1 $\gets$ #2}
\algrenewcommand\algorithmicrequire{\textbf{Input:}}
\algrenewcommand\algorithmicensure{\textbf{Input:}}
\newcommand\numberthis{\addtocounter{equation}{1}\tag{\theequation}}

\newcommand{\epi}{\mathrm{epi}}
\newcommand{\Rea}{{\Bbb R}}
\newcommand{\Int}{{\Bbb Z}}
\newcommand{\Rat}{{\Bbb Q}}
\newcommand{\Cmp}{{\Bbb C}}
\newcommand{\Nat}{{\Bbb N}}
\newcommand{\NN}{\mathbb{N}}
\newcommand{\RR}{\mathbb{R}}
\newcommand{\oRR}{\overline{\RR}}
\newcommand{\argmax}{\mathrm{argmax}}
\newcommand{\argmin}{\mathrm{argmin}}
\newcommand{\interior}{\mathrm{int}}
\newcommand{\Fix}{\mathrm{Fix}}
\newcommand{\dom}{\mathrm{dom}}
\newcommand{\graph}{\mathrm{gph}}
\DeclarePairedDelimiter{\dotp}{\langle}{\rangle}


\setlength{\oddsidemargin}{0in}
\setlength{\evensidemargin}{0in}
\setlength{\textwidth}{6.5in}
\setlength{\topmargin}{-.25in}
\setlength{\textheight}{8.5in}
%\setlength{\voffset}{0in}
%\setlength{\headheight}{0in}
%\setlength{\headsep}{0in}
%\setlength{\topskip}{0in}

\newtheorem{definition}{Definition}
\newtheorem{remark}{Remark}
\newtheorem{theorem}{Theorem}
\newtheorem{lemma}[theorem]{Lemma}
\newtheorem{corollary}[theorem]{Corollary}
\newtheorem{proposition}[theorem]{Proposition}
\newtheorem{claim}[theorem]{Claim}
\newtheorem{observation}{Observation}
\newtheorem{fact}{Fact}
\newtheorem{example}{Example}
\newtheorem{exercise}{Exercise}
\newtheorem{notation}{Notation}




\newenvironment{proof}{\noindent{\bf Proof:} \hspace*{1em}}{
    \hspace*{\fill} $\Box$ }
\newenvironment{proof_of}[1]{\noindent {\bf Proof of #1:}
    \hspace*{1em} }{\hspace*{\fill} $\Box$ }
\newenvironment{proof_claim}{\begin{quotation} \noindent}{
    \hspace*{\fill} $\diamond$ \end{quotation}}


\newcommand{\handout}[5]{
   \renewcommand{\thepage}{#1-\arabic{page}}
   \noindent
   \begin{center}
   \framebox{
      \vbox{
    \hbox to 6.5in { {\bf ORIE 6300 Mathematical Programming I} \hfill #2 }
       \vspace{4mm}
       \hbox to 6.5in { {\Large \hfill #5  \hfill} }
       \vspace{2mm}
       \hbox to 6.5in { {\it #3 \hfill #4} }
      }
   }
   \end{center}
   \vspace*{4mm}
}

\newcommand{\lecture}[4]{\handout{#1}{#2}{Lecturer:
#3}{Scribe: #4}{Lecture #1}}
\newcommand{\problemset}[4]{\handout{#1}{#2}{}{Due Date: #4}{Problem Set #3}}
\newcommand{\problemsetsoln}[3]{\handout{#1}{#2}{}{}{Problem Set #3 Solutions}}
\newcommand{\exam}[3]{\handout{#1}{#2}{}{Due Date: #3}{Take-Home Final Exam}}
\newcommand{\examsoln}[2]{\handout{#1}{#2}{}{}{Take-Home Final Exam Solutions}}
\newcommand{\recitation}[4]{\handout{#1}{#2}{Lecturer:
#3}{Topic: #4}{Recitation #1}}


\newcommand{\dpw}{David P.\ Williamson}

\newenvironment{alglist}{\begin{list}{}{\setlength{\leftmargin}{1.5cm}
\setlength{\rightmargin}{0cm}\setlength{\itemsep}{1ex}\setlength{\parsep}{1ex}}}{\end{list}}

\newcommand{\problem}[3]
{\fbox{\parbox{6in}{{\bf #1}\begin{itemize}\item{\bf Input:} {#2} \item{\bf Goal:} {#3}\end{itemize}}}}
%\newcommand{\algorithm}[2]
%{\fbox{\fbox{\begin{minipage}{6in}
%{\bf #1}\vspace*{.1cm}\hrule
%\begin{tabbing}
%99\qquad\=\qquad\=\qquad\=\qquad\=\qquad\=\qquad\=\kill
%#2
%\end{tabbing}
%\end{minipage}}}}
%../../preamble}

\usepackage{array}
\usepackage{latexsym}
\usepackage{theorem}
\usepackage{amsmath}
\usepackage{amssymb}
\usepackage{xspace}

\usepackage{graphicx}

\newcommand{\reals}{\mathbb{R}}

\newcommand{\scf}{\ensuremath{\mathcal{F}}\xspace}
\newcommand{\scc}{\ensuremath{\mathcal{C}}\xspace}

\begin{document}

\recitation{2}{August 31, 2016}{Calvin Wylie}{Facility
Location}

\section*{Facility Location\footnote{Based on previous notes of Chaoxu Tong}}

\subsection*{Problem Definition}

In the facility location problem, we have a set of facilities $\scf$
and a set of clients $\scc$. The clients all have some demands that
must be met. The facilities can service those demands, but only if
they are open. Since the clients will always connect to the
``closest'' open facility, opening many facilities will allow
clients to connect cheaply. However, there is also a cost associated
with opening facilities, so opening fewer facilities at the cost of
longer connections may be beneficial.

\begin{figure}[h]
\centerline{
\includegraphics{FacilityLocation1.pdf}
}
\end{figure}

Formally, each facility $i \in \scf$ has a cost $f_i$ which is
incurred if facility $i$ is opened. For every facility $i$ and
client $j$, there is a connection cost $c_{ij}\geq0$. Each client $j$ must
connect to a single facility $i$ at cost $c_{ij}$. A solution is
a subset $S$ of the facilities $\scf$ which will be opened, and the
cost of this solution is

$$ \sum_{i \in S} f_i + \sum_{j \in \scc} \left( \min_{i \in S} c_{ij} \right).$$

We want to find a solution with minimal cost. %We assume that the
%connection costs obey the triangle inequality ($c_{ik} \leq c_{ij} +
%c_{jk} \hspace{12pt} \forall i,j,k \in \scf \cup \scc$). The
%examples in this recitation will deal with connection costs given by
%Euclidean distance in $\reals^2$, which has this property.

\subsection*{IP Formulation}

Consider indicator ($\{0,1\}$) variables $y_i$, $i\in \scf$ and
$x_{ij}$, $(i,j)\in \scf\times\scc$ as representing whether we open
facility $i$ or whether we assign client $j$ to facility $i$,
respectively. This leads to a natural integer programming model
given by:\\
\\
$\begin{array}{lrrll} \min & \sum_{i\in\scf} f_iy_i& +
\sum_{(i,j)\in\scf\times\scc}c_{i,j}x_{i,j}&&\\
\mbox{subject to:}& &\sum_{i\in\scf} x_{i,j}&=1&\forall j\in\scc\\
& y_i & -x_{i,j}&\geq 0 & \forall (i,j)\in\scf\times\scc\\
& &y_i\in\{0,1\}&& \forall i\in\scf\\
& &x_{i,j} \in\{0,1\} && \forall (i,j)\in\scf\times\scc\\
\end{array}$\\
\\
The constraint $\sum_{i\in\scf} x_{i,j}=1$ represents that every
client must be assigned to some facility while the constraint $y_i
-x_{i,j}\geq 0$ represents that client $j$ can only be assigned to
facility $i$ if we actually open facility $i$.


\subsection*{Calculating Lower Bounds}
Just as with linear programs, we are often interested in calculating
lower bounds on the objective value of integer programs. In this
section, we will consider a scheme for deriving a family of lower
bounds.\\
\\
Initially, assume that all facility costs $f_i = 0$, i.e., there is
no cost with opening a facility. Then, each client will choose to
connect with the facility that has the lowest cost associated with
it; hence, the effective cost for client $j$, denoted $v_j :=
\min_{i\in\scf} c_{i,j}$, and the total objective cost is simply
$\sum_{j\in\scc} v_j$. Clearly, this is a lower bound on the optimal
 IP value.\\
\\
Now, let's generalize this a little. Suppose that, instead of
incurring the entire cost of opening a facility, each client has to
pay specifically for the portion of a facility's resources it uses.
More specifically, for each facility $i$, there are non-negative
costs $w_{i,j}$ such that if client $j$ is assigned to facility $i$,
it must also pay $w_{i,j}$. Additionally, assume $\sum_{j\in\scc}
w_{i,j}\leq f_i$, i.e., the total of these individual costs is no
more than the total
cost of opening the facility.\\
\\
To arrive at a lower bound, note that each client will want to
minimize his total cost. Hence, each client's effective cost is now
$v_j = \min_{i\in\scf} \lbrace c_{i,j}+w_{i,j} \rbrace$. Once again, the
corresponding lower bound on the objective cost is
$\sum_{j\in\scc}v_j$.\\
\\
First, we can relax $v_j = \min_{i\in\scf} \lbrace c_{i,j}+w_{i,j} \rbrace$ to $v_j
\leq c_{i,j}+w_{i,j}\ \forall (i,j)\in\scf\times\scc$. Next, note
that we never specified the actual costs, just simply that
$\sum_{j\in\scc} w_{i,j}\leq f_i$. So, maximize over the $w_{i,j}$
variables to get the best possible lower bound. This leads to an LP
given by:\\
\\
$\begin{array}{lrrll}
\max & \sum_{j\in\scc} v_j &&&\\
\mbox{s.t.} & v_j & - w_{i,j} & \leq c_{i,j} & \forall
(i,j)\in\scf\times\scc\\
&&\sum_{j\in\scc} w_{i,j} & \leq f_i & \forall i\in\scf\\
&&w_{i,j}&\geq 0&\forall
(i,j)\in\scf\times\scc\\
\end{array}$
\\

So, the problem of coming up with lower bounds for the facility
location has been reduced to solving a linear program. By now, upon
seeing a linear program, the first natural question to ask is what
the dual represents.\\
\\
Noting that there are $|\scf\times\scc|$ constraints in the first
constraint set of the LP and $|\scf|$ in the second, let $x_{i,j}$
and $y_i$ be the corresponding dual variables. Then, the
corresponding dual LP is :\\
\\

$\begin{array}{lrrll} \min & \sum_{(i,j)\in\scf\times\scc}
c_{i,j}x_{i,j}& + \sum_{i\in\scf}f_iy_i &&\\
\mbox{s.t.}& \sum_{i\in\scf}x_{i,j}&&=1&\forall j\in\scc\\
& -x_{i,j}&+y_i&\geq 0&\forall (i,j)\in\scf\times\scc\\
&x_{i,j}\geq 0& y_i\geq 0&& (i,j)\in\scf\times\scc\\
\end{array}$\\
\\
Note that this is exactly the LP relaxation of the original facility
location problem. Hence, the method of combinatorially constructing
lower bounds for our original problem corresponds directly with
relaxing the integer program into a linear program and solving the
dual.

\subsection*{Taking the Dual of an LP}

\noindent\begin{tabular}{ll}
\\
Primal LP:\\
max & $c^Tx$\\
such that: & $Ax = b$ \\
 & $x \geq 0$ \\
\\
Dual LP:\\
min & $b^Ty$\\
such that: & $A^Ty \geq c$ \\
\\
\end{tabular}

Let's review the elements of dual construction. The dual is
constructed to create a bound, so that the optimal value of the dual
is greater than the optimal value of the primal. Proving this weak
duality requires nothing more than matrix multiplication for some
feasible $x$ and $y$:

$$ c^Tx \leq y^TAx = y^Tb $$

Here, we have the first inequality because $x \geq 0$ and $A^Ty \geq
c$, and the second equality because $Ax = b$.
\\
\\
However, if for some row $i$ of constraint $A$, we used the
constraint $\sum_{j=1}^n a_{ij} x \leq b_i$, we would need an
inequality for our variable $y_i$. Using $y_i \geq 0$ would
guarantee that the direction of the inequality is maintained. On the
other hand, if our primal used the constraint $\sum_{j=1}^n a_{ij} x
\geq b_i$, we would need a different requirement for $y_i$.
Specifically, $y_i \leq 0$ would ensure that the inequality flips so
that the constraint $\sum_{j=1}^n a_{ij} x \geq b_i$ reverses
direction, and can upper bound the primal LP value. Similarly, if we
have some $x_j \leq 0$, then we require that for column $j$ of $A$,
$\sum_{i=1}^m a_{ij} y\leq c_j$ for the same reason. If $x_j$ was
unconstrained, we would instead require $\sum_{i=1}^m a_{ij} y =
c_j$ because the sign of $x_j$ is unknown.
\\
\\
This gives us the rules for nonstandard dual construction. There is
a variable in the dual corresponding to each constraint in the
primal, and there is a constraint in the dual corresponding to each
variable in the primal. The relationships are as follows:
\\
\\
\begin{tabular}{lll}
Primal (maximization) & & Dual (minimization) \\
\\
Constraint $\sum_{j=1}^n a_{ij} x \leq b_i$ & $\Rightarrow$ & Variable $y_i \geq 0$\\
Constraint $\sum_{j=1}^n a_{ij} x \geq b_i$ & $\Rightarrow$ & Variable $y_i \leq 0$\\
Constraint $\sum_{j=1}^n a_{ij} x = b_i$ & $\Rightarrow$ & Variable $y_i$ unconstrained\\
\\
Variable $x_j \geq 0$ & $\Rightarrow$ & Constraint $\sum_{i=1}^m
a_{ij} y \geq c_j$\\
Variable $x_j \leq 0$ & $\Rightarrow$ & Constraint $\sum_{i=1}^m
a_{ij} y \leq c_j$\\
Variable $x_j$ unconstrained & $\Rightarrow$ & Constraint
$\sum_{i=1}^m a_{ij} y = c_j$\\
\end{tabular}
\\
\\





\end{document}
