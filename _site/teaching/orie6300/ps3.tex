\documentclass[11pt]{article}

\usepackage{mathtools}

\usepackage[noend]{algpseudocode}
\usepackage{algorithm}
\usepackage{algorithmicx}
\newcommand*\Let[2]{\State #1 $\gets$ #2}
\algrenewcommand\algorithmicrequire{\textbf{Input:}}
\algrenewcommand\algorithmicensure{\textbf{Input:}}
\newcommand\numberthis{\addtocounter{equation}{1}\tag{\theequation}}

\newcommand{\epi}{\mathrm{epi}}
\newcommand{\Rea}{{\Bbb R}}
\newcommand{\Int}{{\Bbb Z}}
\newcommand{\Rat}{{\Bbb Q}}
\newcommand{\Cmp}{{\Bbb C}}
\newcommand{\Nat}{{\Bbb N}}
\newcommand{\NN}{\mathbb{N}}
\newcommand{\RR}{\mathbb{R}}
\newcommand{\oRR}{\overline{\RR}}
\newcommand{\argmax}{\mathrm{argmax}}
\newcommand{\argmin}{\mathrm{argmin}}
\newcommand{\interior}{\mathrm{int}}
\newcommand{\Fix}{\mathrm{Fix}}
\newcommand{\dom}{\mathrm{dom}}
\newcommand{\graph}{\mathrm{gph}}
\DeclarePairedDelimiter{\dotp}{\langle}{\rangle}


\setlength{\oddsidemargin}{0in}
\setlength{\evensidemargin}{0in}
\setlength{\textwidth}{6.5in}
\setlength{\topmargin}{-.25in}
\setlength{\textheight}{8.5in}
%\setlength{\voffset}{0in}
%\setlength{\headheight}{0in}
%\setlength{\headsep}{0in}
%\setlength{\topskip}{0in}

\newtheorem{definition}{Definition}
\newtheorem{remark}{Remark}
\newtheorem{theorem}{Theorem}
\newtheorem{lemma}[theorem]{Lemma}
\newtheorem{corollary}[theorem]{Corollary}
\newtheorem{proposition}[theorem]{Proposition}
\newtheorem{claim}[theorem]{Claim}
\newtheorem{observation}{Observation}
\newtheorem{fact}{Fact}
\newtheorem{example}{Example}
\newtheorem{exercise}{Exercise}
\newtheorem{notation}{Notation}




\newenvironment{proof}{\noindent{\bf Proof:} \hspace*{1em}}{
    \hspace*{\fill} $\Box$ }
\newenvironment{proof_of}[1]{\noindent {\bf Proof of #1:}
    \hspace*{1em} }{\hspace*{\fill} $\Box$ }
\newenvironment{proof_claim}{\begin{quotation} \noindent}{
    \hspace*{\fill} $\diamond$ \end{quotation}}


\newcommand{\handout}[5]{
   \renewcommand{\thepage}{#1-\arabic{page}}
   \noindent
   \begin{center}
   \framebox{
      \vbox{
    \hbox to 6.5in { {\bf ORIE 6300 Mathematical Programming I} \hfill #2 }
       \vspace{4mm}
       \hbox to 6.5in { {\Large \hfill #5  \hfill} }
       \vspace{2mm}
       \hbox to 6.5in { {\it #3 \hfill #4} }
      }
   }
   \end{center}
   \vspace*{4mm}
}

\newcommand{\lecture}[4]{\handout{#1}{#2}{Lecturer:
#3}{Scribe: #4}{Lecture #1}}
\newcommand{\problemset}[4]{\handout{#1}{#2}{}{Due Date: #4}{Problem Set #3}}
\newcommand{\problemsetsoln}[3]{\handout{#1}{#2}{}{}{Problem Set #3 Solutions}}
\newcommand{\exam}[3]{\handout{#1}{#2}{}{Due Date: #3}{Take-Home Final Exam}}
\newcommand{\examsoln}[2]{\handout{#1}{#2}{}{}{Take-Home Final Exam Solutions}}
\newcommand{\recitation}[4]{\handout{#1}{#2}{Lecturer:
#3}{Topic: #4}{Recitation #1}}


\newcommand{\dpw}{David P.\ Williamson}

\newenvironment{alglist}{\begin{list}{}{\setlength{\leftmargin}{1.5cm}
\setlength{\rightmargin}{0cm}\setlength{\itemsep}{1ex}\setlength{\parsep}{1ex}}}{\end{list}}

\newcommand{\problem}[3]
{\fbox{\parbox{6in}{{\bf #1}\begin{itemize}\item{\bf Input:} {#2} \item{\bf Goal:} {#3}\end{itemize}}}}
%\newcommand{\algorithm}[2]
%{\fbox{\fbox{\begin{minipage}{6in}
%{\bf #1}\vspace*{.1cm}\hrule
%\begin{tabbing}
%99\qquad\=\qquad\=\qquad\=\qquad\=\qquad\=\qquad\=\kill
%#2
%\end{tabbing}
%\end{minipage}}}}

\usepackage{amsmath, latexsym, epsf}
\newcommand{\RR}{\mathbf{R}}
\newcommand{\epi}{\textbf{epi}}
\begin{document}

\problemset{3}{September 8, 2016}{3}{September 15, 2016}

\begin{enumerate}
\item \textbf{Alternatives.} The Theorem of Alternatives (proved in lecture 6) states that for a given matrix $A \in \RR^{m\times n}$ and a given vector $b \in \RR^m$, exactly one of the following two statements holds:
\begin{enumerate}
\item $\exists x \in \RR^n$ such that $ Ax \leq b$.
\item $\exists y \in \RR^m$ such that $y \geq 0$, $A^T y = 0$, and $b^T y < 0$.
\end{enumerate}
Use the Theorem of Alternatives to prove
\begin{itemize}
\item \textbf{Farkas' Lemma:} Given a matrix $A \in \RR^{ m\times n}$ and $b \in \RR^m$, exactly one of the following hold:
\begin{itemize}
\item $\exists x \in \RR^n$ such that $Ax = b$ and $x \geq 0$.
\item $\exists y \in \RR^m$ such that $A^T y \geq 0$ and $b^T y < 0$.
\end{itemize}
\item \textbf{Mixed Theorem of Alternatives.} Given $A \in \RR^{ m_1 \times n}$, $C \in \RR^{m_2 \times n}$, and $b \in \RR^{m_1}, d \in \RR^{m_2}$, exactly one of the following hold:
\begin{itemize}
\item $\exists x\in \RR^n$ such that $Ax \leq b$ and $Cx = d$.
\item $\exists z \in\RR^{m_1}, y \in \RR^{m_2}$ such that $z \geq 0$, $A^Tz + C^Ty = 0$, and $b^T z + d^Ty < 0$.
\end{itemize}
\end{itemize}
\item \textbf{Normal Cones and Strong Duality.} Given $A, \overline{A}, \hat{A}, \widetilde{A}, b, \overline{b}, c, \overline{c}$:
\begin{enumerate}
\item for all $z = \begin{bmatrix} x & \overline{x}\end{bmatrix}^T$, compute $N_{P}(z)$, where 
$$
P = \left\{\begin{bmatrix} x \\ \overline{x}\end{bmatrix} \mid Ax + \overline{A}\overline{x} \leq b, \hat{A}x + \widetilde{A}\overline{x} = \overline{b}, x \geq 0\right\}. 
$$
 \item for all $w = \begin{bmatrix} y & \overline{y}\end{bmatrix}^T$, compute $N_{D}(w)$, where 
$$
D = \left\{\begin{bmatrix} y \\ \overline{y}\end{bmatrix} \mid A^Ty + \hat{A}^T\overline{y} \geq  c, \overline{A}^Ty + \widetilde{A}^T\overline{y} = \overline{c}, y \geq 0\right\}. 
$$
\item Using the descriptions of the above normal cones, and \textbf{without using the strong duality theorem we proved in class}, prove the following variant of the strong duality theorem: suppose that both of the linear programs 
\begin{align*}
\max_{\begin{bmatrix} x \\ \overline{x}\end{bmatrix} \in P}c^Tx + \overline{c}^T \overline{x} && && \max_{\begin{bmatrix} y \\ \overline{y}\end{bmatrix} \in D}b^Ty + \overline{b}^T \overline{y}
\end{align*}
have finite optimal values, which we denote by $p^\ast$ and $d^\ast$, respectively. Then $p^\ast = d^\ast.$ 
\end{enumerate}
\item \textbf{Subgradients.} Let $f : \RR^n \rightarrow \RR$ be a continuous, convex function, i.e., $f$ satisfies
\begin{align*}
\left(\forall \lambda \in [0, 1]\right)\left(\forall x, y \in \RR^n\right)\quad f(\lambda x + (1-\lambda)y ) \leq \lambda f(x) + (1-\lambda) f(y).
\end{align*}
Define the \textit{epigraph} of $f$: 
$$
\epi(f) := \{ (x, t) \in \RR^{n+1} \mid f(x) \leq t\}.
$$
\begin{itemize}
\item Prove that $\epi(f)$ is closed and convex. 
\item Prove that $\begin{bmatrix} v & -1 \end{bmatrix}^T \in N_{\epi(f)}(x, f(x))$ if, and only if, 
\begin{align*}
\left(\forall z \in \RR^n \right) \quad f(z) \geq f(x) + (z - x)^Tv.
\end{align*}
Such vectors $v \in \RR^n$ are called \textit{subgradients $f$ at $x$}, and the set of all such $v$ is denoted by $\partial f(x)$.
\item Prove Fermat's rule: $x$ minimizes $f$ if, and only if, $ 0 \in \partial f(x)$.
\item Let $f : \RR \rightarrow \RR$ be the absolute value function $f(x) = |x|$. Draw the following normal cones: $N_{\epi(f)}(-1, 1), N_{\epi(f)}(0, 0), $ and $N_{\epi(f)}(1, 1)$.
\end{itemize}
\item \textbf{Vertices of Dual Problems.} 
\begin{itemize} 
\item Prove the following lemma: consider the polyhedron $Q:= \{x \mid x \geq 0, Cx = d\}$. Then
\begin{itemize}
\item $\overline{x}\in Q$ is an extreme point if 
\begin{align*}
\text{rank}\left(\begin{bmatrix} c_{i_1} & c_{i_2}& \ldots & c_{i_k} \end{bmatrix}\right) = k
\end{align*}
where $c_j$ is column $j$ of $C$ and $\{i_1,i_2, \ldots, i_k\} = \{i \mid \overline{x}_i > 0\}.$
\item any extreme point of $Q$ has at most $\text{rank}(C)$ nonzero elements.
\end{itemize}
\item Suppose that the following linear program has a solution:
\begin{align*}
\text{minimize} &\;  b^T y  \\
\text{subject to:}&\; A^Ty = c\\
&\; y \geq 0
\end{align*}
Using Problem 4 from homework 2, conclude that at least one solution to this  problem is a vertex. Use the first part of this exercise to show that there exists a solution $y^\ast$ with at most $\text{rank}(A^T)$ nonzero elements. 
\item Using part 2, prove \textbf{Carath{\'e}odory's Theorem}: if $x \in \RR^n$ is the convex combination of $k$ vectors, $v_1, \ldots, v_k$, then $x$ is also the convex combination of at most $n+1$ of the vectors $v_1, \ldots, v_k$.
\end{itemize}
\end{enumerate}
\end{document}
