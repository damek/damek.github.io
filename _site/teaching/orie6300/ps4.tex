\documentclass[11pt]{article}

\usepackage{mathtools}

\usepackage[noend]{algpseudocode}
\usepackage{algorithm}
\usepackage{algorithmicx}
\newcommand*\Let[2]{\State #1 $\gets$ #2}
\algrenewcommand\algorithmicrequire{\textbf{Input:}}
\algrenewcommand\algorithmicensure{\textbf{Input:}}
\newcommand\numberthis{\addtocounter{equation}{1}\tag{\theequation}}

\newcommand{\epi}{\mathrm{epi}}
\newcommand{\Rea}{{\Bbb R}}
\newcommand{\Int}{{\Bbb Z}}
\newcommand{\Rat}{{\Bbb Q}}
\newcommand{\Cmp}{{\Bbb C}}
\newcommand{\Nat}{{\Bbb N}}
\newcommand{\NN}{\mathbb{N}}
\newcommand{\RR}{\mathbb{R}}
\newcommand{\oRR}{\overline{\RR}}
\newcommand{\argmax}{\mathrm{argmax}}
\newcommand{\argmin}{\mathrm{argmin}}
\newcommand{\interior}{\mathrm{int}}
\newcommand{\Fix}{\mathrm{Fix}}
\newcommand{\dom}{\mathrm{dom}}
\newcommand{\graph}{\mathrm{gph}}
\DeclarePairedDelimiter{\dotp}{\langle}{\rangle}


\setlength{\oddsidemargin}{0in}
\setlength{\evensidemargin}{0in}
\setlength{\textwidth}{6.5in}
\setlength{\topmargin}{-.25in}
\setlength{\textheight}{8.5in}
%\setlength{\voffset}{0in}
%\setlength{\headheight}{0in}
%\setlength{\headsep}{0in}
%\setlength{\topskip}{0in}

\newtheorem{definition}{Definition}
\newtheorem{remark}{Remark}
\newtheorem{theorem}{Theorem}
\newtheorem{lemma}[theorem]{Lemma}
\newtheorem{corollary}[theorem]{Corollary}
\newtheorem{proposition}[theorem]{Proposition}
\newtheorem{claim}[theorem]{Claim}
\newtheorem{observation}{Observation}
\newtheorem{fact}{Fact}
\newtheorem{example}{Example}
\newtheorem{exercise}{Exercise}
\newtheorem{notation}{Notation}




\newenvironment{proof}{\noindent{\bf Proof:} \hspace*{1em}}{
    \hspace*{\fill} $\Box$ }
\newenvironment{proof_of}[1]{\noindent {\bf Proof of #1:}
    \hspace*{1em} }{\hspace*{\fill} $\Box$ }
\newenvironment{proof_claim}{\begin{quotation} \noindent}{
    \hspace*{\fill} $\diamond$ \end{quotation}}


\newcommand{\handout}[5]{
   \renewcommand{\thepage}{#1-\arabic{page}}
   \noindent
   \begin{center}
   \framebox{
      \vbox{
    \hbox to 6.5in { {\bf ORIE 6300 Mathematical Programming I} \hfill #2 }
       \vspace{4mm}
       \hbox to 6.5in { {\Large \hfill #5  \hfill} }
       \vspace{2mm}
       \hbox to 6.5in { {\it #3 \hfill #4} }
      }
   }
   \end{center}
   \vspace*{4mm}
}

\newcommand{\lecture}[4]{\handout{#1}{#2}{Lecturer:
#3}{Scribe: #4}{Lecture #1}}
\newcommand{\problemset}[4]{\handout{#1}{#2}{}{Due Date: #4}{Problem Set #3}}
\newcommand{\problemsetsoln}[3]{\handout{#1}{#2}{}{}{Problem Set #3 Solutions}}
\newcommand{\exam}[3]{\handout{#1}{#2}{}{Due Date: #3}{Take-Home Final Exam}}
\newcommand{\examsoln}[2]{\handout{#1}{#2}{}{}{Take-Home Final Exam Solutions}}
\newcommand{\recitation}[4]{\handout{#1}{#2}{Lecturer:
#3}{Topic: #4}{Recitation #1}}


\newcommand{\dpw}{David P.\ Williamson}

\newenvironment{alglist}{\begin{list}{}{\setlength{\leftmargin}{1.5cm}
\setlength{\rightmargin}{0cm}\setlength{\itemsep}{1ex}\setlength{\parsep}{1ex}}}{\end{list}}

\newcommand{\problem}[3]
{\fbox{\parbox{6in}{{\bf #1}\begin{itemize}\item{\bf Input:} {#2} \item{\bf Goal:} {#3}\end{itemize}}}}
%\newcommand{\algorithm}[2]
%{\fbox{\fbox{\begin{minipage}{6in}
%{\bf #1}\vspace*{.1cm}\hrule
%\begin{tabbing}
%99\qquad\=\qquad\=\qquad\=\qquad\=\qquad\=\qquad\=\kill
%#2
%\end{tabbing}
%\end{minipage}}}}

\usepackage{amsmath, latexsym,amssymb}

\begin{document}


\problemset{4}{September 15, 2016}{4}{September 22, 2016}

\begin{enumerate}

%\item In class we showed that given a polytope $Q = conv(v_1,\ldots,v_k)$, if 0 is in the interior of $Q$, then $Q$ is a bounded polyhedron.
%Now suppose that we only know that there is some point $v$ in the interior of $Q$.  Show that $Q$ is bounded polyhedron.  (Hint:  Think about $Q-v = \{w-v: w \in Q\}$).
\item Compute the projection operator $P_S$ for each of the following closed, convex sets $S$:
\begin{enumerate}
\item $S = \{x \in \RR^n \mid x \geq 0\}$.
\item $S = [-1, 1]^n$.
\item $S = \{x \mid Ax = b\}$ where $A \in \RR^{m \times n}$ and $b \in \RR^m$.
\item $S = \{x \mid a^Tx \leq b\}$ where $a \in \RR^n $ and $b \in \RR$.
\end{enumerate}

%\item Let $A \in \RR^{m \times n}$ be a matrix, let $b \in \RR^m$ be a vector, and let $c \in \RR^n$ be a vector. Define the nonlinear operator 
%\begin{align*}
%T : &\; \RR^n \times \RR^m \rightarrow \RR^n \times \RR^m\\
%T
%\begin{bmatrix} 
%x \\
%y
%\end{bmatrix}
% &=\begin{bmatrix} 
% P_{\{x \mid Ax \leq b\}}\left(x + c - A^Ty\right) \\ 
% P_{\{y \mid A^Ty = c, y \geq 0\}}\max\{y + Ax - b, 0\}
%\end{bmatrix},
%\end{align*}
%where $\max\{y + Ax - b, 0\} \in \RR^m$ denotes the component-wise max, and for any closed, convex set $S$, $P_S$ denotes the projection operator of $S$. Consider the primal and dual pair of linear programs:
%\begin{align*}
%\max\{ c^Tx \mid Ax \leq b\} && \text{and} && \min\{b^T y \mid A^T y= c, y \geq 0\}.
%\end{align*}
%\begin{enumerate}
%\item Prove that $z := \begin{bmatrix} 
%x &y
%\end{bmatrix}^T$ is a fixed point of $T$ (i.e., $Tz = z$) if, and only if, $x$ is primal-optimal, $y$ is dual-optimal, and $c^Tx = b^Ty$.
%\item Let $C$ be a nonempty, closed, convex set. Prove that the projection map $P_{C}$ is 1-Lipschitz continuous, i.e., 
%$$
%\left(\forall x \in \RR^n\right), \left(\forall y \in \RR^n\right) \quad \|P_Cx - P_Cy\| \leq \|x - y\|.
%$$ (Hint: recall that $\left(\forall x \in \RR^n \right) \; x - P_C(x) \in N_C(P_C(x))$.)
%\item Recall the Brouwer fixed-point theorem: 
%\begin{theorem}
%Let $K \subseteq R^k$ be a compact, convex set, and let $f :K \rightarrow K$ be continuous. Then there exists a point $z \in K$ such that $f(z) = z$.
%\end{theorem}
%Suppose that $\{x \mid A x \leq b\}$ and $ \{y \mid A^T y = c, y \geq 0\}$ are nonempty and bounded. Prove that $T$ has a fixed point.
%\item Use the above results to prove von-Neumann's minimax theorem: let $B \in \RR^{m \times n}$, and for all $k \in \NN$ define $\Delta_k = \{z \in \RR^k \mid z \geq 0, \sum_{i=1}^k z_i = 1\}$. Then we have
%\begin{align*}
%\min_{x \in \Delta_n} \left(\max_{y \in \Delta_m} y^T B x\right) = \max_{y \in \Delta_m} \left(\min_{x \in \Delta_n} y^T B x\right).
%\end{align*}
%\end{enumerate}

\item Consider the set $P=\{x: A x \ge 0\}$ and assume that we have $x \ge 0$
for all $x \in P$, i.e., that $x \ge 0$ is implied by $ A x \ge 0$.
\begin{enumerate}
\item A set $K$ is a cone if $x,y\in K$ implies that $\lambda x + \mu y \in K$
for all $\mu, \lambda \ge 0$. Prove that $P$ is a cone.
\item An {\em extreme ray} of a cone $K$ is a nonzero vector $x \in K$ such
that $x+y \in K$ and $x-y \in K$ implies that $y=\lambda x$ for some
$\lambda$.

Give another characterization of the extreme rays of the polyhedral cone $P$,
using the rank of a submatrix of $A$. (Hint: think about the positive orthant
as the canonical example of a cone, in order to get some intuition here.)

\item Two extreme rays $x$ and $y$ of a cone $K$ are said to be the same if
$x=\lambda y$ for some $\lambda >0$.  Prove that the number of different
extreme rays of our polyhedral cone $P$ is finite. Give a finite bound on
the maximum number of extreme rays possible assuming that $A$ is has $m$ rows
and $n$ columns.
\item Let $r^1$, \ldots, $r^k$ denote the finite set of extreme rays of
$P$. Let
$$Q= cone(r^1,\ldots,r^k)=
\{x=\sum_i \lambda_i r^i: \lambda_i \ge 0 \mbox{ for all $i$}\}.
$$
Prove that $P=Q$. (Hint: consider $P'=\{x \in P: \sum x_i=1\}$. )

It might help to visualize this as moving from the description of
$P$ by the faces of the cone that bound it ($Ax \ge 0$) to a
description of $P$ by the outside rays ($r^1, \ldots, r^k$) that
bound it.
\end{enumerate}


\item (Strict Complementary Slackness)  Consider the standard form linear programs, with primal LP $(\min c^Tx: Ax = b, x \geq 0)$ and dual LP $(\max b^Ty: A^Ty \leq c)$.  Suppose the value of the two LPs is $\gamma$.
\begin{enumerate}
\item Show that the set of optimal solutions to the primal is a convex set; argue the same for the dual.
\item Show that either there exists an optimal solution $x$ to the primal such that $x_j > 0$ or there exists an optimal solution $y$ to the dual such that the $j$th inequality is strict; that is,  $\sum_{i=1}^na_{ij}y_i < c_j.$  (Hint: Consider the linear program $(\min -e_j^Tx: Ax = b, -c^Tx \geq -\gamma, x \geq 0)$, where $e^j$ is a vector that has a 1 in the $j$th component, and 0 everywhere else).
\item  Show that there exist a primal optimal solution $x^*$ and a dual optimal solution $y^*$ such that for all $j$, $x^*_j > 0$ if and only if the $j$th inequality of the dual is met with equality.
\end{enumerate}

    

\end{enumerate}
\end{document}
